%-------------------------
% Resume in Latex
% Author : Radhakrishnan Ravi Vignesh
% License : MIT
%------------------------

\documentclass[letterpaper,11pt]{article}

\usepackage{latexsym}
\usepackage[empty]{fullpage}
\usepackage{titlesec}
\usepackage{marvosym}
\usepackage[usenames,dvipsnames]{color}
\usepackage{verbatim}
\usepackage{enumitem}
\usepackage[pdftex]{hyperref}
\usepackage{fancyhdr}
\usepackage{ragged2e}

\pagestyle{fancy}
\fancyhf{} % clear all header and footer fields
\fancyfoot{}
\renewcommand{\headrulewidth}{0pt}
\renewcommand{\footrulewidth}{0pt}

% Adjust margins
\addtolength{\oddsidemargin}{-0.375in}
\addtolength{\evensidemargin}{-0.375in}
\addtolength{\textwidth}{1in}
\addtolength{\topmargin}{-.5in}
\addtolength{\textheight}{1.0in}

\urlstyle{same}

\raggedbottom
\raggedright
\setlength{\tabcolsep}{0in}

% Sections formatting
\titleformat{\section}{
  \vspace{-4pt}\scshape\raggedright\large
}{}{0em}{}[\color{black}\titlerule \vspace{-5pt}]

%-------------------------
% Custom commands
\newcommand{\resumeItem}[2]{
  \item\small{
    \textbf{#1}{: #2 \vspace{-2pt}}
  }
}
\newcommand{\resitem}[2]{
  \item\small{
    \textbf{#1}{: #2 \vspace{-8 pt}}
  }
}
\newcommand{\ritem}[1]{
  \item\small{
    {#1 \vspace{-2pt}}
  }
}

\newcommand{\rit}[1]{
  \item\small{
    {#1 \vspace{-6pt}}
  }
}

\newcommand{\resumeSubheading}[4]{
  \vspace{-1pt}\item
    \begin{tabular*}{0.97\textwidth}{l@{\extracolsep{\fill}}r}
      \textbf{#1} & #2 \\
      \textit{#3} & \textit{#4} \\
    \end{tabular*}\vspace{-5pt}
}

\newcommand{\resumeSubItem}[2]{\resumeItem{#1}{#2}\vspace{-4pt}}

\renewcommand{\labelitemii}{$\circ$}


\newcommand{\resumeexperiencestart}{\begin{itemize}[leftmargin=*]}
\newcommand{\resumeexperienceend}{\end{itemize}}
\newcommand{\resumeSubHeadingListStart}{\begin{description}[leftmargin=*]}
\newcommand{\resumeSubHeadingListEnd}{\end{description}}
\newcommand{\resumeItemListStart}{\begin{itemize}[leftmargin=*]}
\newcommand{\resumeItemListEnd}{\end{itemize}\vspace{-4pt}}

%-------------------------------------------
%%%%%%  CV STARTS HERE  %%%%%%%%%%%%%%%%%%%%%%%%%%%%


\begin{document}

%----------HEADING-----------------
\begin{tabular*}{\textwidth}{l@{\extracolsep{\fill}}r}
  \textbf{\Large Radhakrishnan Ravi Vignesh}\\
  \href{https://www.github.com/vigneshr97}{Github: https://www.github.com/vigneshr97} & Mobile: +1-979-676-6684 \\
  \href{https://www.linkedin.com/in/vigneshr97}{Linkedin: https://www.linkedin.com/in/vigneshr97} & Email: \href{mailto:vigneshr97@gmail.com}{vigneshr97@gmail.com}
\end{tabular*}\vspace{-8pt}

%-----------EDUCATION-----------------
\section{Education}
  \resumeSubHeadingListStart
    \resumeSubheading
      {Texas A\&M University}{College Station, TX}
      {Master of Science in Civil Engineering;  Specialization: Transportation Engineering}{Aug. 2018 -- Present}
    \resumeSubheading
      {Indian Institute of Technology Madras}{Chennai, India}
      {B.Tech in Civil Engineering; Minor: Industrial Engineering; CGPA: 8.01/10.00}{Aug. 2014 -- July. 2018}
  \resumeSubHeadingListEnd

%-----------INTERESTS------------------
\section{Interests}
   \resumeSubHeadingListStart
    \ritem
      {Network Optimization, Computer Vision, Deep Learning}
  \resumeSubHeadingListEnd
  
%-----------SKILLS------------------
\section{Relevant Technical Skills}
   \resumeSubHeadingListStart
    \resumeSubItem{Programming Languages}
      {C, C++, Python, R}
    \resumeSubItem{Libraries}
      {Numpy, OpenCV, Scikit Learn, Pandas, Tensorflow, Keras, Eigen(C++)}
    \resumeSubItem{Software Skills}
      {AutoCAD, Revit, MATLAB, Civil 3D, \LaTeX , git}
  \resumeSubHeadingListEnd

%-----------EXPERIENCE-----------------
\section{Experience}
  \resumeSubHeadingListStart
    \resumeSubheading
      {Bachelor’s Thesis: Optimization of Network Algorithms}{IIT Madras}
      {Advisor: Dr. Karthik K Srinivasan}{Aug 2017 - June 2018}
      \resumeItemListStart
        \justifying\ritem{Optimized the \textbf{label setting algorithm} given by D.Shier for finding the \textbf{best K elementary paths} in a directed graph and bench marked against other best known algorithms such as that of \href{http://www.dis.uniroma1.it/challenge9/papers/pascoal.pdf}{Marta M.B Pascoal} and Hoffman \& Pavley.}
        \ritem{Experiments were performed on \textbf{randomized grid networks} and real world networks and compared with other existing algorithms. The algorithm performed significantly better than other algorithms \textbf{asymptotically}.}
        \justifying\ritem{The algorithm was used to arrive at \textbf{Pareto-optimal} solutions for multi-objective optimization problem}
      \resumeItemListEnd
      
    \resumeSubheading
      {Tamil Nadu Health Systems Project)}{Government of Tamil Nadu, India}
      {Software Developer Intern}{May 2017 - July 2017}
      \resumeItemListStart
        \ritem
          {Developed a \textbf{heuristic algorithm} with an accuracy of over \textbf{90\%} to find the victim pickup locations of emergency vehicles and hence automating the data entry of victim pickup time and hospital reach time of the vehicles.}
        \ritem
          {The \textbf{zonal distribution} of calls was used for \textbf{greedy allocation} of ambulances optimizing the scene arrival time.}
      \resumeItemListEnd

  \resumeSubHeadingListEnd


%-----------PROJECTS-----------------
\section{Projects}
  \resumeItemListStart
   \justifying \resumeSubItem{Behavioral Cloning}
      {Trained a simulator to drive a car around a track smoothly. An \textbf{image regression} model was developed after performing various modifications on NVIDIA's deep neural network architecture using \textbf{Transfer learning} technique. \textbf{Keras} library was used in the implementation.}
    \justifying\resumeSubItem{Traffic Sign Classifier}
      {Developed a traffic sign classifier model using deep neural networks with an architecture similar to \textbf{LeNet} using Tensorflow library and obtained \textbf{94\%} test accuracy using German Traffic sign data set. Various \textbf{data augmentation} techniques were employed to arrive at a higher accuracy.}
    \justifying\resumeSubItem{Lane Detection}
      {The lane region and the radii of curvature of the lane boundaries were detected in a video recorded from a car. The video involved various brightness and shadow levels along with different textures on the road.}
    \justifying\resumeSubItem{Extended Kalman Filters}
      {The measurements from a noisy Laser and a noisy Radar were \textbf{fused} using Kalman Filter and Extended Kalman Filter Equations and the position and velocity of objects were detected.}
  \resumeItemListEnd

%-----------COURSEWORK----------------
\section{Coursework}
  \resumeSubHeadingListStart
    \resumeSubItem{Texas A\&M}
    {Urban Transportation Planning, Traffic Engineering: Characteristics, Street and Highway Design}
    \justifying\resumeSubItem
      {IIT Madras}{Transportation Network Analysis, Analytical Techniques in Transportation Engineering, Fundamentals of Operations Research, Industrial Engineering, Computer Simulation, Computer Applications in Transportation Engineering, Decision Modelling, Probability and Statistics, Differential Equations}
    \resumeSubItem
      {Udacity Self Driving Car Engineer Nanodegree}{Computer Vision, Deep Learning, Sensor Fusion}
  \resumeSubHeadingListEnd


%-----------EXTRA CURRICULAR ACTIVITIES-----------------
\section{Extra Curricular Activities}
  \resumeItemListStart
    \justifying\rit{\textbf{Captained} IIT Madras Squash Team during 2017-18 including an annual Pan India Sports Meet and also led the team to a silver medal finish in the Inter Collegiate Sports Fest held in Chennai, India}
    \justifying\rit{Active Volunteer of National Services Scheme of India - IIT Madras Chapter, 2014-15 and taught math to disadvantaged rural children and hence was awarded the \textbf{Star Volunteer Award}}
  \resumeItemListEnd

%-------------------------------------------
\end{document}
